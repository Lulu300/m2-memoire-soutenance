\documentclass[main.tex]{subfiles}

\section{Sélection}

\begin{frame}{Comment choisir ?}

\begin{itemize}
\item Analyse des cas d'utilisation.
\item Définition de critères de choix.
\item L'approche de sélection en entreprise.
\end{itemize}

\end{frame}

\begin{frame}{Architecture}

\begin{table}[h!]
    \centering
	\begin{tabular}{|p{8cm}|p{2.5cm}|} 
  	\hline
  	\textbf{Critère} & \textbf{Architecture} \\
  	\hline
  	Prédiction d'évènement entrant à l'aide de modèle d'apprentissage automatique & Lambda\\
  	\hline
  	Traitement des données en temps réel et par lots radicalement différents & lambda \\
  	\hline
  	Traitement des données par lots complexe & lambda \\
  	\hline
  	Très faible latence entre récupération et affichage des données & Kappa \\
  	\hline
  	Traitement des données par lots et en temps réel similaires & Kappa \\
  	\hline
  	Stockage permanent des données batch avant le traitement & Lambda/Kappa \\
  	\hline
	\end{tabular}
    \caption{Table des critères pour le choix de l'architecture}
    \label{tab:critere-arch}
\end{table}

\end{frame}

\begin{frame}{Message Broker}

\begin{table}[h!]
    \centering
	\begin{tabular}{|p{8cm}|p{2.5cm}|} 
  	\hline
  	\textbf{Critère} & \textbf{Solution} \\
  	\hline
  	Garantir la consommation d'un message par un seul consommateur & RabbitMQ\\
  	\hline
  	Nécessité d'ingérer rapidement une grande quantité de messages & Kafka\\
  	\hline
  	Ordre des messages primordial & Kafka\\
  	\hline
  	Nécessité de conserver les messages à plus au moins long terme & Kafka\\
  	\hline
  	Utilisation de protocoles spécifiques (MQTT, AMQP, ...) & RabbitMQ / ActiveMQ\\
  	\hline
  	Règles de routage des messages complexe & RabbitMQ / ActiveMQ\\
  	\hline
	\end{tabular}
    \caption{Table des critères pour le choix du logiciel d'agent de messages}
    \label{tab:critere-message-broker}
\end{table}

\end{frame}

\begin{frame}{Ingestion des données}

\begin{table}[h!]
    \centering
	\begin{tabular}{|p{8cm}|p{2.5cm}|} 
  	\hline
  	\textbf{Critère} & \textbf{Solution} \\
  	\hline
  	Manque de connaissances pour la réalisation de programmes personnalisés & ETL/ELT\\
  	\hline
  	Nécessité d'extraire de nombreuses sources de données & ETL/ELT\\
  	\hline
  	Peu de sources de données & Programme personnalisé\\
  	\hline
  	Nécessité d'avoir des performances élevées & Programme personnalisé\\
  	\hline
	\end{tabular}
    \caption{Table des critères pour le choix d'une solution complète ou d'un programme personnalisé pour l'ingestion des données.}
    \label{tab:critere-etl-programme}
\end{table}

\end{frame}

\begin{frame}{Traitement des données : Batch}

\begin{table}[h!]
    \centering
	\begin{tabular}{|p{8cm}|p{2.5cm}|} 
  	\hline
  	\textbf{Critère} & \textbf{Solution} \\
  	\hline
  	Nécessité d'utiliser des librairies autres que l'apprentissage automatique & Spark\\
  	\hline
  	Nécessité d'avoir des performances accrues & Spark\\
  	\hline
  	Nécessité d'avoir une tolérance à la panne exemplaire & MapReduce\\
  	\hline
  	Les performances ne sont pas la priorité & MapReduce\\
  	\hline
	\end{tabular}
    \caption{Table des critères pour le choix de la solution de traitement par lots.}
    \label{tab:critere-batch}
\end{table}

\end{frame}

\begin{frame}{Traitement des données : Temps réel}

\begin{table}[h!]
    \centering
	\begin{tabular}{|p{8cm}|p{2.5cm}|} 
  	\hline
  	\textbf{Critère} & \textbf{Solution} \\
  	\hline
  	Source de données en micro batch & Spark Streaming\\
  	\hline
  	Source de données en streaming & apaches Storm\\
  	\hline
	\end{tabular}
    \caption{Table des critères pour le choix de la solution de traitement en temps réel.}
    \label{tab:critere-real-time}
\end{table}

\end{frame}

\begin{frame}{Stockage des données}

\begin{itemize}
\item Clés/Valeurs.
\item Grandes colonnes.
\item Séries temporelles.
\item Orientée graphe.
\item Orientée documents.
\item Moteur d'indexation.
\end{itemize}

\end{frame}

\begin{frame}{Visualisation et analyse des données}

\begin{table}[h!]
    \centering
	\begin{tabular}{|p{8cm}|p{2.5cm}|} 
  	\hline
  	\textbf{Critère} & \textbf{Solution} \\
  	\hline
  	Monitoring & Grafana\\
  	\hline
  	Visualisation complexe & D3.js\\
  	\hline
  	Visualisation simple & Kibana/Solr\\
  	\hline
	\end{tabular}
    \caption{Table des critères pour le choix de la solution de visualisation et analyse des données.}
    \label{tab:critere-viz}
\end{table}

\end{frame}

\begin{frame}{Orchestration des données}

\begin{table}[h!]
    \centering
	\begin{tabular}{|p{8cm}|p{2.5cm}|} 
  	\hline
  	\textbf{Critère} & \textbf{Solution} \\
  	\hline
  	Nécessité d'effectuer une action spécifique en cas d'erreur & Apache Oozie\\
  	\hline
  	Nécessité de lancer une succession de tâche & Apache Oozie\\
  	\hline
  	Exécution de tâche simple & Cron\\
  	\hline
	\end{tabular}
    \caption{Table des critères pour le choix de la d'orchestration.}
    \label{tab:critere-orchestration}
\end{table}

\end{frame}
